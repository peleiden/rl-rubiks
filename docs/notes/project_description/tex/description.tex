% !TeX spellcheck = da_DK
\documentclass[12pt,fleqn,]{article}

\usepackage[danish]{babel}
\usepackage{../../../report/texfiles/SpeedyGonzales}
\usepackage{../../../report/texfiles/MediocreMike}


\title{Projektbeskrivelse til løsning af Rubiks terning med reinforcement learning}
\author{Søren, Anne og Asger}
\date{\today}

\fancypagestyle{plain}
{
	\fancyhf{}
	\rfoot{Side \thepage{} af \pageref{LastPage}}
	\renewcommand{\headrulewidth}{0pt}
}
\pagestyle{fancy}
\fancyhf{}
\lhead{}
\chead{}
\rhead{}
\rfoot{Side \thepage{} af \pageref{LastPage}}


\linespread{1.15} 

\begin{document}

\maketitle
\noindent
Rubiks terning er et eksempel på et diskret kombinatorisk problem med et utroligt stort tilstandsrum på $4.33\times 10^{19}$ distinkte, lovlige konfigurationer og kun én løsning. 
Projektet vil forsøge at løse dette problem ved hjælp af reinforcement learning. 

Den nyeste litteratur har formået at løse dette problem ved at kombinere dyb læring med approksimativ værdiiteration og vægtet A*-søgning\footnote{https://www.nature.com/articles/s42256-019-0070-z}.
Intentionen med projektet er at forsøge selv at implementere dele af denne model. 
Med tid som en begrænsende faktor vil vi dog forsimple problemet, således at vi løser terningekonfigurationer, som kun er få skridt fra løsningen.
Derudover vil vi benytte gruppeteori til at beskrive udvalgte teoretiske aspekter af problemet.
\\\\
De konkrete mål for projektet er at
\begin{itemize}
	\item implementere en effektiv intern repræsentation af Rubiks terning
	\item benytte gruppeteori til at beskrive udvalgte aspekter af problemet
	\item implementere træsøgning på Rubiks terning
	\item implementere dyb Q-læring herunder et neuralt netværk med todelt struktur og eventuelt residuel arkitektur
	\item at samle træsøgningsmetoder og dyb Q-læring til at implementere model der er baseret på DeepCubeA 
	\item at kunne løse lette terningekonfigurationer
\end{itemize}
Implementationen kan bruges til at afprøve og udvikle nye metoder til andre lignende problemer, også med store tilstandsrum, eksempelvis molekylestrukturer og solcelleoptimering. 



 

\end{document}

















